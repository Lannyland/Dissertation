\documentabstract{
When working with a complex AI or robotics system in a specific application, users often need to incorporate their special domain knowledge into the autonomous system. Such needs call for the ability to \textit{manage autonomy}. However, managing autonomy can be a difficult task because the internal mechanisms and algorithms of the autonomous components may be beyond the users' understanding. We propose an approach where users manage autonomy indirectly by \textit{managing information} provided to the intelligent system hierarchically at three different scales: Strategic, Between-Episodes, and Within-Episode. Information management tools at multiple scales allow users to influence the autonomous behaviors of the system without the need for tedious direct/manual control. Information fed to the system can be in the forms of areas of focus, representations of task difficulty, and the amount of autonomy desired. We apply this approach to using an Unmanned Aerial Vehicle (UAV) to support Wilderness Search and Rescue (WiSAR). This dissertation presents autonomous algorithms/components and autonomy management tools/interfaces we designed at different scales, and show that this approach improves the performance of the human-robot team and the experience of the human partner.
}