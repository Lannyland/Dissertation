%=====================================================================================================
%=====================================================================================================
\chapter{Complexity Analysis of the UAV Path Planning Problem}
\label{chap:complexity}

In this appendix, we analyze why an heuristic approach is preferred than a dynamic programming / reinforcement learning approach by comparing the computational complexity of each approach.

%=====================================================================================================
\section{Computational Complexity of Dynamic Programming}
\label{DPComplexity}

The key issue is that the path planning problem that we are solving cannot be solved in polynomial time using dynamic programming unless P=NP\footnote{The path planning problem we are solving is similar to the Orienteering Problem (OP), which can be seen as a combination between the Knapsack Problem (KP) and the Traveling Salesman Problem (TSP). In a fully connected graph where each vertex has a certain score (prize), with fixed starting vertex and end vertex, the OP problem asks for the path that would achieve the highest score within a given time frame. That is why the OP problem is also called the Prize-Collecting TSP problem and TSP with profits. Scores are entirely additive and each vertex can only be visited once (not all vertices have to visited). OP, like TSP, is a NP-hard problem.

Our path planning problem is different from the OP problem in the following aspects: (a) we use a grid representation because this is compatible with UAV flight, so the search is on a graph that is not fully connected; (b) the Bayesian sensing of the UAVs sensors require that the path planning be able to visit the same vertex repeatedly; and (c) this means that the ``prize'' rewarded for each visit to a vertex is only partially collected and is path dependent.}. The justification for this is as follows:

\begin{itemize}
\item The problem that we are solving is at least as computationally complex as what is known as the Orienteering Problem.  
\item The Orienteering Problem is at least as computationally complex as the Traveling Salesperson Problem.
\item The Traveling Salesperson Problem is in the complexity class f-NP, with its corresponding decision problem in the class NP-complete.
\item By reduction, this means that our path planning problem is NP-hard and cannot be solved in polynomial time unless P=NP.
\item Because the path-planning problem is NP-hard, we cannot solve it using dynamic programming in real time for the planning lengths that we consider (up to 900 planning steps).
\end{itemize}


We have attempted to formulate a specific solution to our path-planning problem using dynamic programming, and performed a complexity analysis of this to demonstrate that, indeed, the complexity of dynamic programming is exponential.  This is included in the appendix.

%=====================================================================================================
\section{Complexity of Reinforcement Learning (Approximate Dynamic Programming)}
\label{RLComplexity}

Reinforcement learning cannot learn an optimal solution to an NP-hard problem in polynomial time\footnote{The approximate dynamic programming / reinforcement learning (ADP/RL) approach does not support (near) real-time solution for OP type problems (For example, Righini and Salanil show in~\cite{Righini2009Decremental} that it takes roughly 1000-3000 seconds to solve an OP type problem with 100 vertices/nodes). The ADP/RL approach also does not scale well due to its complexity. Vansteenwegen et al\. surveyed different approaches to solving the OP~\cite{Vansteenwegen2011Orienteering}. Most of the approaches were heuristic approaches, and the only ADP/RL approach mentioned is~\cite{Righini2009Decremental}.}. If it could, then P=NP. Moreover, reinforcement learning typically requires many, many iterations to reach convergence even for moderately sized problems, meaning it is likely to be significantly slower.  Even dynamic programming-based approaches to reinforcement learning, like learning the transition model and applying policy iteration, cannot run in polynomial time on an exponential problem.

In addition to this fundamental limitation of what reinforcement learning can theoretically do, there is a second practical problem with using reinforcement learning for this problem. This practical limitation is that reinforcement learning approaches tend to get stuck in local minima when there are multiple rewards in the problem.  Indeed, the literature includes many papers that seek to resolve this problem by doing what the reviewer suggests --- trading off exploration and exploitation.  These approaches work in practice for some problems, but not for problems that are exponentially complex.  

Moreover, the state space of our path-planning problem grows exponentially because of the possibility of revisiting states. For each revisit, a new reward function must be defined because the Bayesian approach allows for partial collection of information.  This means that we end up with an exponentially hard problem with an exponentially large state space and a unique reward for each element of the state space.  There is no known reinforcement learning algorithm that can solve such problems, let alone solve it in real-time.

%=====================================================================================================
\section{Classical Dynamic Programming for Our UAV Path Planning Problem}
\label{CDPUAV}

Classical dynamic programming (DP) method is a method for solving complex problems by breaking them down into simpler subproblems. Solution to subproblems can be stored to trade space for time. Problems that can be solved by DP must exhibit two key attributes: optimal substructure and overlapping subproblems. For example, DP method can be used to find the exact solution for the TSP problem with few nodes. However, since the TSP problem is NP-hard, it cannot be solved in polynomial time, unless P=NP. Therefore, even with the DP method, the complexity is still $O(2^n n^2)$. DP method suffers the ``curses of dimensionality'' and does not scale well with complex problems.

Our path planning problem can be reduced to the OP, therefore is also NP-hard. And our path planning problem has a state space of 10,000 nodes and a flight path of 900 (possibly higher in real application) time steps, meaning that theoretically the same node could be visited 900 times. If we treat each visit to the same node as a separate node, the state space expands to 9,000,000 nodes, and tracking the connectivity of all these nodes (not fully connected) also becomes intractable.

In order to support real WiSAR operations, we need to have the path created within seconds. Also in practical Wilderness Search and Rescue scenarios, the search area could be much bigger than the 10,000 nodes we demonstrated. The UAV flight time can also be much longer depending on the type of UAV platforms used. That's why we chose a heuristic approximation approach in solving this problem. The complexity of our approach is $O(n)$ once we have the Mode Goodness Ratio (MGR) heuristic, where $n$ is the flight duration in time steps ($n$=900 in our scenarios). This means our approach is very fast and scales very nicely with the NP-hard problem.


%=====================================================================================================
\section{Reinforcement Learning (Approximate Dynamic Programming) for Our UAV Path Planning Problem}
\label{RLUAV}

Instead of solving for the exact solution, approximate dynamic programming / reinforcement learning (ADP/RL) are approximate methods to search for solutions that approximate the optimal solution for complex problems to avoid the ``curses of dimensionality''. ADP/RL methods have four main sub-elements: a policy, a reward function (immediate payoff), a value function (long-term payoff), and optionally, a model of the environment. A policy defines the learning agent's behavior at a given time, a reward function defines the goal and indicates what is good in an immediate sense, a value function specifies what is good in the long run, and, the model of the environment mimics the behavior of the environment. The idea is to learn the optimal policy iteratively for each state, balancing exploration and exploitation. ADP/RL methods use Markov Decision Process (MDP) and can work with problems that have uncertainty in transition.

In our path planning problem, a node can be visited multiple times, and because our Bayesian approach allows for partial collection of information, the score/prize collected for each visit is different. The reward function and the value function both become path dependent, the state space becomes exponentially large. As described in the previous section, we need real time solutions that scale well when search area and flight duration expand.

%=====================================================================================================
%=====================================================================================================
\chapter{Full Experiment Results for Chapter 5}
\label{chap:result}

Here we present the full experiment results of the four WiSAR scenarios described in Chapter 5. For each scenario, we generate paths with three flight durations ($T=300$, $T=600$, and $T=900$) and compare algorithm computation speed (in seconds) and path $\mathit{Efficiency_{LB}}$ (in \%) for BA, LHC-GW-CONV, Top2, and TopN algorithms (including Top2 and TopN algorithms where $k=5$ and $N=3$). . All numbers shown are averages of 10 runs. Best performance results are displayed in bold font face. Standard deviation ($\sigma$) are also shown for both computation speed and path $\mathit{Efficiency_{LB}}$.

%=============
% TestCase
%=============
\begin{center}
\begin{table*}
{
%\small
\scriptsize
\hfill{}
\setlength{\extrarowheight}{1.5pt}
\begin{tabular}
{|l|c|c|c|c|c|c|c|c|c|c|c|c|}
\hline
& \multicolumn{4}{|c|}{$T=300$} & \multicolumn{4}{|c|}{$T=600$} & \multicolumn{4}{|c|}{$T=900$} \\ 
\hline
& Speed & $\sigma$ & $\mathit{E_{LB}}$ & $\sigma$ & Speed & $\sigma$ & $\mathit{E_{LB}}$ & $\sigma$ & Speed & $\sigma$ & $\mathit{E_{LB}}$ & $\sigma$\\ 
\hline
BA & - & - & 27.59 & - & - & - & 43.54 & - & - & - & 59.56 & - \\ 
\hline
LHC-GW-CONV & 0.17 & 0.01 & 92.26 & 0.00 & 0.33 & 0.07 & 92.68 & 0.00 & 0.51 & 0.09 & 94.03 & 0.01 \\ 
\hline
Top2 (1 layer) & 0.12 & 0.05 & 87.49 & 0.03 & 0.14 & 0.04 & 91.66 & 0.04 & 0.15 & 0.04 & 91.02 & 0.03 \\ 
\hline
TopN (1 layer) & \textbf{0.10} & 0.05 & 91.28 & 0.02 & \textbf{0.08} & 0.05 & 91.93 & 0.04 & \textbf{0.07} & 0.03 & 95.24 & 0.01 \\ 
\hline
Top2 (Hierarchy) & 0.37 & 0.10 & 90.85 & 0.02 & 0.42 & 0.10 & 93.83 & 0.02 & 0.48 & 0.10 & 93.59 & 0.01  \\ 
\hline
TopN (Hierarchy) & 0.91 & 0.21 & \textbf{92.27} & 0.00 & 0.84 & 0.16 & \textbf{95.50} & 0.01 & 0.93 & 0.21 & \textbf{95.56} & 0.01 \\ 
\hline
\end{tabular}}
\medskip
\caption{Algorithms speed and $\mathit{Efficiency_{LB}}$ comparison for the multi-modal synthetic scenario.}
\label{TestCaseTable}
%\vspace*{-5ex}
\end{table*}
\end{center}
Table~\ref{TestCaseTable} shows the experiment results for the synthetic WiSAR scenario with a multi-modal distribution of the missing person location and a simple task difficulty map with three difficulty levels (as shown in Fig.\ref{SyntheticCase}). The UAV path starts from a subregion with high task difficulty (lower right corner).

%=============
% HikerPaul
%=============
\begin{center}
\begin{table*}
{
%\small
\scriptsize
\hfill{}
\setlength{\extrarowheight}{1.5pt}
\begin{tabular}
{|l|c|c|c|c|c|c|c|c|c|c|c|c|}
\hline
& \multicolumn{4}{|c|}{$T=300$} & \multicolumn{4}{|c|}{$T=600$} & \multicolumn{4}{|c|}{$T=900$} \\ 
\hline
& Speed & $\sigma$ & $\mathit{E_{LB}}$ & $\sigma$ & Speed & $\sigma$ & $\mathit{E_{LB}}$ & $\sigma$ & Speed & $\sigma$ & $\mathit{E_{LB}}$ & $\sigma$\\ 
\hline
BA & - & - & 56.95 & - & - & - & 60.07 & - & - & - & 57.11 & - \\ 
\hline
LHC-GW-CONV & 0.30 & 0.16 & 60.18 & 0.13 & 0.47 & 0.03 & 56.76 & 0.00 & 0.98 & 0.16 & 55.18 & 0.00\\ 
\hline
Top2 (1 layer) & \textbf{0.24} & 0.06 & 66.68 & 0.09 & 0.30 & 0.11 & 65.21 & 0.07 & 0.41 & 0.20 & 66.08 & 0.07\\ 
\hline
TopN (1 layer) & 0.25 & 0.07 & 76.19 & 0.08 & \textbf{0.24} & 0.11 & 71.02 & 0.04 & \textbf{0.22} & 0.09 & 68.26 & 0.04\\ 
\hline
Top2 (Hierarchy) & 0.73 & 0.11 & 78.67 & 0.03 & 0.84 & 0.14 & 73.81 & 0.04 & 1.19 & 0.36 & 72.75 & 0.02\\ 
\hline
TopN (Hierarchy) & 1.52 & 0.15 & \textbf{81.43} & 0.03 & 1.73 & 0.25 & \textbf{75.48} & 0.02 & 1.68 & 0.26 & \textbf{74.13} & 0.02\\ 
\hline
\end{tabular}}
\medskip
\caption{Algorithms speed and $\mathit{Efficiency_{LB}}$ comparison for the HikerPaul scenario.}
\label{HikerPaulTable}
%\vspace*{-5ex}
\end{table*}
\end{center}
Table~\ref{HikerPaulTable} shows the experiment results for the HikerPaul WiSAR scenario, in which an elderly couple was reported missing near the Grayson Highlands State Park in Virginia. Fig.\ref{HikerPaulMaps} shows the probability distribution map and the task difficulty map for the scenario. Fig.\ref{HikerPaulPaths} shows example paths generated. Each UAV path starts from the Last Known Position (LKP), which is in the middle of the search region.

%=============
% NewYork53
%=============
\begin{center}
\begin{table*}
{
%\small
\scriptsize
\hfill{}
\setlength{\extrarowheight}{1.5pt}
\begin{tabular}
{|l|c|c|c|c|c|c|c|c|c|c|c|c|}
\hline
& \multicolumn{4}{|c|}{$T=300$} & \multicolumn{4}{|c|}{$T=600$} & \multicolumn{4}{|c|}{$T=900$} \\ 
\hline
& Speed & $\sigma$ & $\mathit{E_{LB}}$ & $\sigma$ & Speed & $\sigma$ & $\mathit{E_{LB}}$ & $\sigma$ & Speed & $\sigma$ & $\mathit{E_{LB}}$ & $\sigma$\\ 
\hline
BA & - & - & 39.95 & - & - & - & 54.27 & - & - & - & 65.08 & - \\ 
\hline
LHC-GW-CONV & \textbf{0.01} & 0.00 & 38.47 & 0.00 & \textbf{0.02} & 0.00 & 56.91 & 0.00 & \textbf{0.02} & 0.00 & 67.38 & 0.00\\ 
\hline
Top2 (1 layer) & 0.75 & 0.15 & 54.42 & 0.04 & 0.92 & 0.60 & 66.61 & 0.03 & 0.81 & 0.55 & 72.79 & 0.02\\ 
\hline
TopN (1 layer) & 0.70 & 0.45 & 59.15 & 0.07 & 0.77 & 0.55 & 68.78 & 0.04 & 0.69 & 0.30 & 74.54 & 0.01\\ 
\hline
Top2 (Hierarchy) & 1.87 & 0.23 & 57.18 & 0.03 & 2.06 & 0.34 & 69.29 & 0.02 & 1.92 & 0.33 & 74.44 & 0.01\\ 
\hline
TopN (Hierarchy) & 5.01 & 0.67 & \textbf{65.39} & 0.03 & 5.76 & 0.96 & \textbf{71.47} & 0.02 & 5.32 & 1.12 & \textbf{77.36} & 0.02\\ 
\hline
\end{tabular}}
\medskip
\caption{Algorithms speed and $\mathit{Efficiency_{LB}}$ comparison for the NewYork53 scenario.}
\label{NewYork53Table}
%\vspace*{-5ex}
\end{table*}
\end{center}
Table~\ref{NewYork53Table} shows the experiment results for the NewYork53 WiSAR scenario, in which a 46 year old male camper was reported missing near Adirondack Park in upperstate New York. Fig.\ref{NewYork53Maps} shows the probability distribution map and the task difficulty map for the scenario. Fig.\ref{NewYork53Paths} shows example paths generated. Each path starts from the Last Known Position (LKP), which is in the middle of the search region.

%=============
% NewYork108
%=============
\begin{center}
\begin{table*}
{
%\small
\scriptsize
\hfill{}
\setlength{\extrarowheight}{1.5pt}
\begin{tabular}
{|l|c|c|c|c|c|c|c|c|c|c|c|c|}
\hline
& \multicolumn{4}{|c|}{$T=300$} & \multicolumn{4}{|c|}{$T=600$} & \multicolumn{4}{|c|}{$T=900$} \\ 
\hline
& Speed & $\sigma$ & $\mathit{E_{LB}}$ & $\sigma$ & Speed & $\sigma$ & $\mathit{E_{LB}}$ & $\sigma$ & Speed & $\sigma$ & $\mathit{E_{LB}}$ & $\sigma$\\ 
\hline
BA & - & - & 39.92 & - & - & - & 45.34 & - & - & - & 49.39 & -\\ 
\hline
LHC-GW-CONV & \textbf{0.01} & 0.00 & 41.38 & 0.00 & \textbf{0.45} & 0.06 & 52.88 & 0.00 & \textbf{0.02} & 0.00 & 52.61 & 0.00\\ 
\hline
Top2 (1 layer) & 0.98 & 0.33 & 58.37 & 0.04 & 0.90 & 0.36 & 54.18 & 0.02 & 1.44 & 0.65 & 57.33 & 0.02\\ 
\hline
TopN (1 layer) & 0.92 & 0.38 & 54.03 & 0.06 & 0.83 & 0.56 & 53.91 & 0.07 & 0.97 & 0.42 & 57.91 & 0.03\\ 
\hline
Top2 (Hierarchy) & 2.42 & 0.39 & \textbf{60.73} & 0.02 & 2.52 & 0.67 & 55.91 & 0.01 & 2.50 & 0.23 & 57.94 & 0.01\\ 
\hline
TopN (Hierarchy) & 6.81 & 1.10 & 59.60 & 0.02 & 6.59 & 0.98 & \textbf{60.26} & 0.01 & 7.42 & 1.11 & \textbf{60.99} & 0.02\\ 
\hline
\end{tabular}}
\medskip
\caption{Algorithms speed and $\mathit{Efficiency_{LB}}$ comparison for the NewYork108 scenario.}
\label{NewYork108Table}
%\vspace*{-5ex}
\end{table*}
\end{center}
Table~\ref{NewYork108Table} shows the experiment results for the NewYork53 WiSAR scenario, in which two teenage female hikers were reported missing near West Chesterfield in Massachusetts. Fig.\ref{NewYork108Maps} shows the probability distribution map and the task difficulty map for the scenario. Fig.\ref{NewYork108Paths} shows example paths generated. Each path starts from the Last Known Position (LKP), which is in the middle of the search region.

%=====================================================================================================
%=====================================================================================================

\chapter{Hierarchical Coarse-to-Fine Search and Hierarchical Decision Making}
\label{chap:hierarchical}

