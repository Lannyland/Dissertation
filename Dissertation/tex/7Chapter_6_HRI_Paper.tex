\chapter[Paper: Sliding Autonomy for UAV Path Planning: Adding New Dimensions to Autonomy Management]{Paper: Sliding Autonomy for UAV Path Planning: Adding New Dimensions to Autonomy Management\footnote {To be submitted to JHRI (Journal of Human-Robot Interaction) journal. Authors are Lanny Lin, and Michael A. Goodrich.}}
\label{chap:JHRI2014}

\begin{abstract}
Increased use of autonomy also increases human-automation interaction and the need for humans to manage autonomy. We propose a new autonomy management approach, sliding autonomy, where the user can influence the behavior of the automous system along two new dimensions: a temporal dimension for how much time to allocate to the subtask, and a spatial dimension for where to set task constraints. We apply the approach to the task of UAV (Unmanned Aerial Vehicle) path planning to support Wilderness Search and Rescue (WiSAR), and evaluate the usefulness of the approach against manual and simple pattern path planning modes in a user study. Results show that the sliding autonomy approach performs significantly better than the other two modes, and the performance of the human-automation team outperforms either human or automation working alone. And the users' mental workload, measured by a secondary task performance and NASA TLX, 
\end{abstract}



