\chapter{Background, Motivation, and Overview}
Simulating non-rigid bodies with data from motion capture is a new application of motion capture. We present a motion reconstruction approach for non-rigid bodies in the context of motion capture.  Passive optical motion capture records movements of non-rigid bodies by tracking locations of retroreflective sensors placed on subjects. By processing the recorded motion data and combining the data with physical and statistical models, we create complete animation of non-rigid bodies. We focus on animating natural trees and ropes as instances of non-rigid bodies.  Motion is captured in the presence of external forces, such as wind and human interaction. 

Rope is one of the simplest non-rigid bodies. Animating ropes is a simple platform from which to evaluate the motion capture process, including data collection and processing. The capture process then can be generalized and applied to other non-rigid subjects, such as natural trees. We present methods that can be used to create natural tree motion in wind using physical and statistical models. The resulting motion is visually plausible and scalable. 

Animating non-rigid bodies has important applications in computer-generated movies, video games, and forestry engineering. In movie and video games where computational resources are a concern, directors need a simple process to use the motion of non-rigid bodies as part of telling a story. Motion capture hardware has been popular in movie and game production for many years.  Motion capture equipment is easy to set up and provides motion data with good accuracy in both time and space.  The hardware integrates camera calibration algorithms and greatly simplifies the process of recording position information in real-world 3D space. Many movies use motion capture to record motion from real characters and then drive virtual characters with the collected motion data \cite{Jesser2010}. Motion capture produces realistic motion data while reducing the cost for creating motion in different environments or in fictional environments. 

The equipment has flexibility for capturing both rigid bodies and non-rigid bodies. However, most research and applications focus on rigid bodies. As a non-rigid body, a tree has motion that is important in film and games because trees swaying in the wind can set the mood or feel for a scene, so directors can use tree motion as a tool for telling stories. In the field of forestry, our research helps forestry engineers understand the influence of wind on trees. This can inform decisions for pruning trees or protecting structures. Rope, the other application explored in this work, has simple structure and dynamics. But when the motion of a human character interacting with the rope is captured in the same scene, the replay of both rope and human motion becomes difficult. Movie makers may be able to use our motion capture process and create animation with interactions between rigid and non-rigid bodies.

Human motion capture using a rigid body model, such as \cite{Lou:EHM2010,Wen:2006:MCD,Rajko:2007:RAK,ZordanVictorBrian2003}, is a well-studied area in motion capture but is less applicable to our research, as removing the rigid body assumption changes the problem. There are two common approaches to extracting human motion from motion capture data. One is to extract a skeleton model or kinematic model from motion data \cite{Lou:EHM2010,Rajko:2007:RAK}. The other is to predefine the skeleton and to apply motion capture-data to animate the predefined skeleton \cite{Wen:2006:MCD,ZordanVictorBrian2003}. Most of the methods have an important assumption that the distance between any pair of markers is non-changeable. However, this assumption does not stand for non-rigid bodies, which is the essential difference for motion capture of rigid bodies and non-rigid bodies. Instead of a skeleton of rigid parts, non-rigid motion capture focuses on reconstructing a mesh that deforms to match a moving surface such as a face or cloth. We investigate non-rigid motion capture but we focus on curved splines rather than curved surfaces.

Motion capture of non-rigid bodies is difficult. Prior work on non-rigid motion capture is mainly focused on facial motion \cite{Lorenzo03,SifakisEftychios2005} and cloth motion \cite{MarcusVolker04clothmotion,whiteRyan2007siggraph,PritchardDavid2003,Bhat2003}. Both facial motion and cloth motion aim to represent the motion of a surface. Motion capture of thin, rod-shaped non-rigid bodies such as rope or tree branches is a fundamentally different problem. For instance, rope, as a non-rigid body, is more naturally represented as a curved spline rather than as a curved plane (though a curved spline can be used to drive the motion of a plane). Worring's \cite{WorringMarcel94measurementof} prior work in reconstruction of a line-shaped object in 3D from several computer images builds non-rigid rope motion but would require a foreground-- background separation step and a sharp image of the rope in each frame.

Part of animating trees in wind is modeling the 3D shape of the tree to be animated. Tree shape modeling has long been studied on computer graphics. A tree with natural appearance greatly improves visual quality of tree animation. These methods include particle systems \cite{Reeves83particlesystems,Runions07,palubicki:siggraph09,neubert:acmtg07}, L-systems \cite{lindenmayer68,Lintermann1999,Prusinkiewicz:2001}, parametric models \cite{Weber1995}, photographs \cite{RecheMartinez2004,neubert:acmtg07,Tan:2007:ITM} or videos \cite{Li:2011:MGM,Diener:2006}. Most of these methods result in satisfying tree shapes but do not leverage the 3D positions of motion capture markers, which are recorded as part of a motion capture session but require a different set of inputs. Image- or video-based approaches convert a set of 2D input images into 3D tree models by filling in the missing dimension.  Motion capture systems can record tree shape in 3D with high precision (using similar techniques for converting a set of 2D images to a 3D model). Prior work \cite{Long:MCN2010} in reconstructing tree shape from motion capture data using either exact measurements or markers placed within the crown does not scale and does not apply when leaves occlude the branching structure.

Animation of trees in wind has been discussed for many years \cite{Akagi:cg06,stams:eu97,shinya:eu92}. Prior work in animating 3D tree models focuses on recreating branch motion due to wind turbulence. Wind turbulence has been simulated and has been synthesized from the frequency spectrum of turbulence created by tree crowns. Simulation-based models create tree motion based on the tree's biomechanical characteristics and wind dynamics \cite{Akagi:cg06,Habel09PGT}. Spectral approximation describes tree swaying and wind velocity field using some computer-generated noise. These systems include techniques based on photographs \cite{wu:cas99} or videos \cite{Diener:2006}, and some parameter-based spectral models \cite{Habel09PGT,shinya:eu92}. In most of the previous research, the wind field is created using noise and fluid simulation. While simulation models capture visually important wind--tree effects, such as crown sheltering, they require expensive computations that are not currently feasible in interactive applications such as games. Spectral approximation ignores sheltering effects and requires significant user intervention, but is computationally efficient. Rather than being based on actual captured tree motion, simulation models and spectral approximations are both theoretically based. Motion capture avoids the directability and computation problems of simulated wind fields but may yield data that validates simulation-based models. 

% From Egbert: put a paragraph here that summarize how your approach is different from the other works that has been done.
Our work differs from the previous work by introducing motion capture in simulating non-rigid bodies. It is also the first work to motion capture bendy thin rods and the first to replay motion capture data from a tree. The resulted animations are mostly driven by data instead of pure physical simulation or stochastic process. We describe a complete approach to animating natural trees using motion capture, including data collection, data processing, generating tree shape, and combining statistics and wind dynamics for the animation. The thesis discusses two approaches for tree animation with different motion capture setups, which require different processes and create tree motion with varying scalability. Rope motion is created using a data-driven approach rather than the pure physical model that is common in previous research. The rope project helps us to better understand the motion-recording capability of motion capture for non-rigid bodies. Using motion capture simplifies the process to parameterize a physical model of a rope under specific motion.

\section{Thesis Statement}

A small portion of known movements collected from motion capture can produce complete movements of non-rigid bodies by combining the motion capture data with physical and statistical models. We show that the resulting motion is visually plausible for trees and rope.

\section{Publications}
This dissertation consists of seven papers, two of which are under review.  The work can be divided into three parts: rope motion reconstruction, tree shape reconstruction, and tree motion reconstruction. The remaining chapters contain these papers.  Chapter 2 is a pilot project in which a simplified motion capture process is presented with statistical analysis for replaying natural tree motion where the motion is not scalable. There is no attempt to retarget or extend the captured data. While we were able to replay the tree motion, it requires a series of careful measurements to recreate the exact tree shape, and the motion cannot be transferred to similar shapes or situations. Chapter 3 describes replaying rope motion. The project focuses on processing motion capture data and uses rope's simple structure to validate estimated data against the collected motion data. The process for motion capture data includes identifying and removing noises and labeling markers. This process was used for both tree and rope motion capture. Chapters 4 and 5 begin a different approach to the problem.  Rather than directly replay the original motion on the original structure, in Chapter 4 and 5 we reconstruct the approximate tree shape from the captured motion in preparation for replaying similar motion on a similar structure. Chapter 6 extracts a force field from captured motion that can be used to animate any tree shape.  

In Chapter 2, the pilot project aims to replay natural tree swaying in wind. Retroreflective markers are placed on a small cherry tree. We infer a skeleton from tree motion data and repair the motion data using a rigid body model. The motion data contain gaps and errors for branches that bend. Motion-data repair is critical because trees are not real rigid bodies. These ideas allow the reconstruction of tree motion, including global effects, but without a complex physical model. Instead, it employs a statistical model. This project builds a complete pipeline for motion capture, including equipment set up, data collecting interface, 3D tree model, and motion integration. However, the scalability of this approach is limited and it requires intensive labor to reproduce the exact tree structure. The work requires markers being placed about 10 cm apart on each branch segment. Leaves have to be removed to enlarge the visibility of markers to motion capture cameras. Branches are assumed to be rigid (to some degree) with small bending capability. 

Chapter 3 begins a study of motion capture for rope and takes a closer look at animating non-rigid bodies using a non-rigid model for motion capture. Rope has a simple structure with no branches and simple geometry. The model is a good test bed for evaluating different algorithms for both data processing and motion replaying. In this project, we provide a more general approach for processing motion capture data from passive optical motion capture for non-rigid bodies rather than adapting a rigid body model as in Chapter 2. The data collection and processing process includes labeling markers and identifying and removing noise. We provide clustering, gap repair, and marker swap detection algorithms based on linear interpolation and forward differencing under the assumption that the rope does not stretch. Indexed marker positions are connected with a spline in each frame to approximate the original rope. The model produces visually plausible animations of rope motion from data collected for a person interacting with rope. However, the method fails when the rope experiences large accelerations that result in motion that is not modeled by forward differencing.

Reconstruction of tree shape, as in Chapters 4 and 5, is an important step in replaying motion capture of trees under external forces, which is the goal of the work presented in Chapter 6. A realistic 3D tree model helps researchers estimate and compare computer-generated animation against the original motion. Existing algorithms for generating branching structures for image synthesis in computer graphics are not adapted to the unique data set provided by motion capture. In Chapter 4, we discuss a method for tree shape reconstruction using particle flow on input data obtained from a passive optical motion capture system. Initial branch tip positions are estimated from averaged and smoothed motion capture data. Branch tips, as particles, are also generated within the bounding space defined by a stack of bounding boxes or a convex hull. The particle flow, starting at branch tips within the bounding volume under forces, creates tree branches. The resulting shapes are realistic and similar to the original tree crown shape. Several tunable parameters provide control flexibility over branching shape and arrangement.

In Chapter 5, we combine a procedural method with particle flow to generate a 3D tree shape. Using L-systems for describing tree branches as particles, our method introduces a hemisphere to generate particles, uses a growth level to simulate different ages of branches, and applies a dynamic bounding box to detect local growth area in a tree. This new method enhances the management of tree shapes by easing the control over distributions of branches and leaves. We also demonstrate that the method has potential to simulate phototropism and growth around physical barriers. We evaluate this model by particle flow and the complexity of this method, showing performance competitive with existing methods.

Chapter 6 discusses non-rigid motion capture by extracting external forces from motion capture data and then replaying those forces to create animation. We explore this idea in the context of motion capture of natural trees in wind. Motion of a tree in wind is decomposed into three forces: wind-induced drag, branch elasticity, and damping by the leaves. Given a model of elasticity and damping, the drag force can be isolated and used to estimate wind velocity. The extracted velocity field is extended to a larger volume and enriched with a turbulence model. That wind field can be replayed on a tree model that includes elastic and damping properties to create similar motion.  The work contained in this chapter is the culminating work of this dissertation.

We list all the citations for each chapter in the order in which they appear.

1. Jie Long, Cory Reimschussel, Ontario Britton, Anthony Hall, and Michael Jones. Performance Capture for Natural Tree Motions in the Wind. Motion in Games , MIG, 2010. 

2. Jie Long, Bryce Porter, Michael Jones. Motion Capture of Rope, not yet published.

3. Jie Long and Michael Jones. 3D Tree Modeling using Motion Capture. IEEE The Fourth International Symposium on Plant Growth Modeling, Simulation, Visualization and Application (PMA '12).

An extended version of this paper is requested for submission to the journal Frontiers in Computer Science . 

4. Jie Long and Michael Jones. A Realistic 3D Tree Model based on L-Systems. Report for UNEP Eco-Peace Leadership Center (EPLC), 2008.

5. Jie Long and Michael Jones. Estimating wind flow from tree motion using motion capture data, not yet published.

In addition, the research on motion capture of trees was presented as a short talk in SIGGRAPH 2009, where a short abstract was published.

Jie Long, Cory Reimschussel, Ontario Britton, and Michael Jones. Motion capture for natural tree animation. International Conference on Computer Graphics and Interactive Techniques, SIGGRAPH 2009: Talks. New Orleans, Louisiana, Article No. 77, 2009. 

Data captured as part of this work was used in a study of tortuosity as a metric for evaluating branch motion paths. The citation for that paper is given below but the paper is not included in the dissertation.  

Michael Jones and Jie Long. Tortuosity as a Metric for Evaluating Branch Motion Paths. IEEE The Fourth International Symposium on Plant Growth Modeling, Simulation, Visualization and Application (PMA '12).