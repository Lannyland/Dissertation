\documentabstract{
Simulating the motion of a non-rigid body under external forces is a difficult problem because of the complexity and flexibility of the non-rigid geometry and its associated dynamics. Physically based animation of objects moving in the wind is computationally expensive, so simulation-based approaches oversimplify the model by ignoring important effects, such as tree's sheltering. Motion capture records actual responses of a non-rigid body to external forces and helps solve these problems. Mainly focusing on natural trees and ropes as instances of non-rigid bodies, we present a new approach to building motion for objects in wind using incomplete motion capture data from non-rigid bodies. The incomplete motion capture data are automatically labeled by a cluster-based algorithm while noises are removed. For places with no motion capture data, we estimate forces and motion by interpolating the motion capture data according to the object's characteristics. We discuss a physically or statistically based approach to animate the whole non-rigid object. Basing our work on the collected motion capture data and the estimated motions, we can produce visually plausible and scalable animations of non-rigid objects under external forces at interactive frame rates. 
}